\documentclass[aspectratio=169]{beamer}
\usetheme{metropolis}

\usepackage{graphicx}
\usepackage{hyperref}
\usepackage{amsmath}
% MGMT 675 Beamer Style
% Include this file in your slide decks with: \input{mgmt675-style}

\usepackage{tcolorbox}
\tcbuselibrary{skins}
\usepackage{tikz}
\usepackage{booktabs}
\usetikzlibrary{shadings}

% Spacing adjustments
\linespread{0.9}  % Tighter baseline skip
\setlength{\parskip}{0.8\baselineskip plus 0.3ex}  % Extra space between paragraphs
\setbeamertemplate{itemize items}[circle]
\setlength{\leftmargini}{1.5em}

% Extra space between bullet items
\let\olditemize\itemize
\renewcommand{\itemize}{%
  \olditemize
  \setlength{\itemsep}{0.7ex}%
  \setlength{\parsep}{0.3ex}%
}

% Define accent color (matches title underline)
\definecolor{accentblue}{RGB}{37,99,235}
% Light blue/teal for shaded boxes
\definecolor{excelinput}{RGB}{232, 244, 247} %{177,214,224}
% Dark teal for titles
\definecolor{titlegray}{RGB}{19,60,71}

% Title slide highlight line color
\definecolor{titlehighlight}{RGB}{9,112,140}
\setbeamercolor{progress bar}{fg=titlehighlight}

% Alert text styling - bold bright coral/orange for contrast
\definecolor{alertorange}{RGB}{230,90,50}
\setbeamercolor{alerted text}{fg=alertorange}
\setbeamerfont{alerted text}{series=\bfseries}

% Hyperlink colors
\hypersetup{
  colorlinks=true,
  linkcolor=titlegray,
  urlcolor=alertorange,
  citecolor=accentblue
}

% Set frame title background
\setbeamercolor{frametitle}{bg=titlegray, fg=white}

% Style block environments with background shading
\setbeamercolor{block title}{fg=white, bg=accentblue}
\setbeamercolor{block body}{fg=black, bg=gray!10}

\setbeamercolor{block title alerted}{fg=white, bg=red!70!black}
\setbeamercolor{block body alerted}{fg=black, bg=red!10}

\setbeamercolor{block title example}{fg=white, bg=green!50!black}
\setbeamercolor{block body example}{fg=black, bg=green!10}

% Theorem environments
\setbeamercolor{theorem text}{fg=black}
\setbeamercolor{theoremblock title}{fg=white, bg=accentblue!80}
\setbeamercolor{theoremblock body}{fg=black, bg=accentblue!10}

% Bulleted list in shaded box
\newenvironment{baritemize}{%
  \begin{tcolorbox}[
    colback=excelinput,
    colframe=accentblue,
    boxrule=0.5pt,
    arc=2mm,
    auto outer arc,
    left=3mm,
    right=3mm,
    top=2mm,
    bottom=2mm
  ]
  \begin{itemize}
}{%
  \end{itemize}
  \end{tcolorbox}
}

% Numbered list in shaded box
\newenvironment{barenumerate}{%
  \begin{tcolorbox}[
    colback=excelinput,
    colframe=accentblue,
    boxrule=0.5pt,
    arc=2mm,
    auto outer arc,
    left=3mm,
    right=3mm,
    top=2mm,
    bottom=2mm
  ]
  \begin{enumerate}
}{%
  \end{enumerate}
  \end{tcolorbox}
}

% Define shaded box environment with slightly rounded corners and dark title bar
\newtcolorbox{shadedbox}[1][]{
  colback=excelinput,
  colframe=accentblue,
  boxrule=0.5pt,
  arc=2mm,
  auto outer arc,
  left=3mm,
  right=3mm,
  top=2mm,
  bottom=2mm,
  after skip=\baselineskip,
  coltitle=white,
  colbacktitle=titlegray,
  fonttitle=\bfseries,
  toptitle=2mm,
  bottomtitle=2mm,
  #1
}

% Gradient background for title slide
\let\oldmaketitle\maketitle
\renewcommand{\maketitle}{%
  {%
    \setbeamertemplate{background}{%
      \begin{tikzpicture}[remember picture,overlay]
        \fill[white] (current page.north west) rectangle (current page.south east);
        \shade[inner color=white, outer color=titlegray!40, shading=radial]
          (current page.north west) circle (28cm);
      \end{tikzpicture}%
    }%
    \oldmaketitle
  }%
}


\title{Software, Rice Database, Technical Indicators}
\subtitle{BUSI 722: Data-Driven Finance II}
\author{Kerry Back}
\institute{}
\date{}

\begin{document}

\maketitle

%%% PART 0: CAPM REVIEW & FAMA-FRENCH %%%

\begin{frame}[c]
\centering
\Huge Review: CAPM \& Fama-French Factors
\end{frame}

\begin{frame}{The CAPM}
The Capital Asset Pricing Model says that the expected excess return of any asset is proportional to the market's expected excess return:
\[
\mathbb{E}[r_i] - r_f \;=\; \beta_i \bigl(\mathbb{E}[r_m] - r_f\bigr)
\]

\begin{baritemize}
\item $r_i$ = return on asset $i$
\item $r_f$ = risk-free rate
\item $r_m$ = return on the market portfolio
\item $\beta_i = \dfrac{\text{Cov}(r_i, r_m)}{\text{Var}(r_m)}$
\end{baritemize}
\end{frame}

\begin{frame}{CAPM $\Leftrightarrow$ Zero Alphas}
Run a time-series regression for each asset $i$:
\[
r_{it} - r_{ft} \;=\; \alpha_i + \beta_i(r_{mt} - r_{ft}) + \varepsilon_{it}
\]

\begin{baritemize}
\item The CAPM is equivalent to saying $\alpha_i = 0$ for every asset.
\item If $\alpha_i > 0$: the asset earns more than the CAPM predicts --- it is ``underpriced.''
\item If $\alpha_i < 0$: the asset earns less than the CAPM predicts --- it is ``overpriced.''
\end{baritemize}

\vspace{0.3cm}
Testing the CAPM $=$ testing whether alphas are jointly zero.
\end{frame}

\begin{frame}{The Size Effect}
\alert{Small stocks have historically earned higher average returns than large stocks}, even after adjusting for beta.

\begin{baritemize}
\item Sort stocks into portfolios by market capitalization
\item Small-stock portfolios earn positive CAPM alphas
\item Large-stock portfolios earn negative CAPM alphas
\item This is evidence against the CAPM
\end{baritemize}

\vspace{0.3cm}
Size is measured by market capitalization $=$ price $\times$ shares outstanding.
\end{frame}

\begin{frame}{The Value Effect}
\alert{Value stocks (high book-to-market) have historically outperformed growth stocks (low book-to-market)}, even after adjusting for beta.

\begin{baritemize}
\item Book-to-market $=$ book equity / market equity
\item High B/M (``value'') stocks: positive CAPM alphas
\item Low B/M (``growth'') stocks: negative CAPM alphas
\item This is further evidence against the CAPM
\end{baritemize}
\end{frame}

\begin{frame}{Fama-French Three-Factor Model (1993)}
Fama and French proposed replacing the CAPM with a three-factor model:
\[
\mathbb{E}[r_i] - r_f \;=\; \beta_i^{\text{MKT}} \cdot \text{MKT} \;+\; \beta_i^{\text{SMB}} \cdot \text{SMB} \;+\; \beta_i^{\text{HML}} \cdot \text{HML}
\]

\begin{baritemize}
\item \textbf{MKT} $= r_m - r_f$: market excess return
\item \textbf{SMB} (Small Minus Big): return on small stocks minus return on large stocks
\item \textbf{HML} (High Minus Low): return on high B/M stocks minus return on low B/M stocks
\end{baritemize}
\end{frame}

\begin{frame}{Interpreting the Three-Factor Model}

The three-factor model says that expected returns are explained by exposures to three sources of risk.

\begin{baritemize}
\item The model ``works'' if alphas are zero when we regress excess returns on the three factors:
\[
r_{it} - r_{ft} = \alpha_i + \beta_i^{\text{MKT}} \text{MKT}_t + \beta_i^{\text{SMB}} \text{SMB}_t + \beta_i^{\text{HML}} \text{HML}_t + \varepsilon_{it}
\]
\item The size and value effects that produced nonzero CAPM alphas are captured by SMB and HML.
\item But \ldots{} other anomalies (momentum, profitability, investment) still produce nonzero alphas.
\end{baritemize}
\end{frame}

%%% PART 1: SETUP %%%

\begin{frame}[c]
\centering
\Huge Setup: Claude Pro \& Claude Code
\end{frame}

\begin{frame}{Initial Steps}

\textbf{1. Sign up for Claude Pro:}

\begin{barenumerate}
\item Visit \href{https://claude.ai}{claude.ai}
\item Click ``Sign Up'' or ``Get Started''
\item Create account using email or Google/Apple sign-in
\item After signing in, click ``Upgrade to Pro'' in the sidebar
\item Choose the monthly payment plan
\item Complete payment information
\end{barenumerate}

2. \href{https://www.youtube.com/watch?v=hpMrTabldEY}{YouTube Video}
\end{frame}

\begin{frame}{Mac -- Install Claude Code}

\textbf{Step 1: Install Node.js}
\begin{itemize}
\item Using Homebrew: \texttt{brew install node}
\item Or download installer from \href{https://nodejs.org}{nodejs.org}
\end{itemize}

\vspace{0.3cm}

\textbf{Step 2: Install Claude Code}
\begin{itemize}
\item Run: \texttt{npm install -g @anthropic-ai/claude-code}
\end{itemize}

\vspace{0.3cm}

\textbf{Step 3: Verify Installation}
\begin{itemize}
\item Run: \texttt{claude doctor}
\end{itemize}
\end{frame}

\begin{frame}{Windows -- Install Claude Code}

\textbf{Step 1: Install Node.js}
\begin{itemize}
\item Download and install from \href{https://nodejs.org}{nodejs.org}
\item Choose the LTS (Long Term Support) version
\end{itemize}

\vspace{0.3cm}

\textbf{Step 2: Install Git for Windows}
\begin{itemize}
\item Download from \href{https://git-scm.com/download/win}{git-scm.com/download/win}
\item During installation, select ``Git Bash'' (default option)
\end{itemize}

\vspace{0.3cm}

\textbf{Step 3: Install Claude Code}
\begin{itemize}
\item In PowerShell, run: \texttt{npm install -g @anthropic-ai/claude-code}
\end{itemize}

\vspace{0.3cm}

\textbf{Step 4: Verify Installation}
\begin{itemize}
\item In PowerShell, run: \texttt{claude doctor}
\end{itemize}
\end{frame}

\begin{frame}{Install Python \& Create Virtual Environment}
\textbf{Install Python:}
\begin{itemize}
\item Tell Claude Code: ``Install Python 3.13 and add it to the path.''
\item Tell Claude Code: ``Upgrade pip''
\end{itemize}

\vspace{0.3cm}

\textbf{Create Virtual Environment:}
\begin{itemize}
\item Tell Claude Code: ``Create a virtual environment using Python 3.13 in my current directory.''
\end{itemize}

\vspace{0.3cm}

\textbf{Install packages in the virtual environment:}
\begin{itemize}
\item \texttt{numpy pandas scipy statsmodels scikit-learn}
\item \texttt{openpyxl matplotlib seaborn ipykernel}
\item \texttt{pandas-datareader streamlit requests python-dotenv}
\end{itemize}
\end{frame}

%%% PART 2: VS CODE %%%

\begin{frame}[c]
\centering
\Huge VS Code \& Database Setup
\end{frame}

\begin{frame}{Install and Open VS Code}
\begin{baritemize}
\item Install VS Code: \href{https://code.visualstudio.com}{code.visualstudio.com}
\item Launch VS Code
\item File $\rightarrow$ Open Folder $\rightarrow$ navigate to your course folder
\item Install extensions: Python, Jupyter, Claude Code, Data Wrangler, Rainbow CSV
\item View $\rightarrow$ Command Palette $\rightarrow$ ``Python: Select Interpreter'' $\rightarrow$ choose venv
\end{baritemize}
\end{frame}

\begin{frame}{Launch Claude Code in VS Code}
\begin{itemize}
\item If you see the orange Claude icon in the top toolbar, click it.
\item If not, create a new file (File $\rightarrow$ New Text File) and you should see it.
\item Or View $\rightarrow$ Command Palette and enter ``Claude Code: Open in New Tab.''
\end{itemize}
\vspace{0.5cm}
Test: Ask Claude Code: What is the sum of the first 1,000 integers?
\end{frame}

\begin{frame}{Stock Market Database}
\begin{itemize}
\item Database Guide: \url{https://portal-guide.rice-business.org}
\item Visit \href{https://data-portal.rice-business.org}{data-portal.rice-business.org} to get an access token
\item The data portal is an AI agent that uses ChatGPT to generate SQL and query the database
\end{itemize}

\vspace{0.3cm}

\textbf{Create .env File:}
\begin{itemize}
\item Tell Claude Code to create a \texttt{.env} file
\item Add: \texttt{RICE\_ACCESS\_TOKEN=your\_token\_here}
\end{itemize}
\end{frame}

\begin{frame}{Data Skills}
Skills are text files that give Claude Code specialized knowledge.

\vspace{0.3cm}

The easiest way to install is to \alert{ask Claude Code to do it for you}: ``Download and install the rice-data-query and merge skills,
plus CLAUDE.md from mgmt638.kerryback.com/skills/''
\end{frame}

\begin{frame}{Workflow}

Visit the Data Guide to find variable names from the SF1 (Sharadar Fundamentals) table.

\vspace{0.3cm}

Tell Claude Code to:
\begin{barenumerate}
\item Get returns (monthly or weekly) from the Rice database.
\item Get the variables you want from SF1.
\item Calculate any ratios or growth rates you want.
\item Merge the returns with the fundamental data.
\end{barenumerate}

\vspace{0.3cm}

Save the merged data.  \alert{You do not need to repeat this process if you already have the data you want.}
\end{frame}

\begin{frame}{Timing of Data}
During merging, Claude will align variables so that
\begin{baritemize}
\item Fundamental variables are all known at the beginning of the period
\item Momentum is momentum as of the beginning of the period
\item Close is the closing price at the end of the previous month or week
\item Return is the return over the period (from beginning to end)
\end{baritemize}

\vspace{0.3cm}

All variables other than return \alert{are known at the beginning of the period} and can be used to pick stocks.
\end{frame}

\begin{frame}{Avoiding Look-Ahead Bias}
\begin{itemize}
\item Price data (\texttt{close}, \texttt{marketcap}, \texttt{pb}) lagged by 1 month
\item Fundamentals available in first full month \emph{after} filing date
\item Forward-filled until next filing
\item Include \textbf{all tickers} (including delisted) to avoid survivorship bias
\end{itemize}
\end{frame}

\begin{frame}{Data Formats}

Claude should automatically save data as parquet files.  This is a compact, fast format.

\vspace{0.3cm}
To view the data, ask Claude to:
\begin{baritemize}
\item \alert{``Convert the file to Excel.''} Then open as usual.
\item \alert{``Convert the file to csv.''} Then double-click in VS Code File Explorer.
\item \alert{``Read the data in a Jupyter notebook.''} Then work with the data in Python.
\end{baritemize}
\end{frame}

\end{document}
