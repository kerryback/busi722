\documentclass[aspectratio=169]{beamer}
\usetheme{metropolis}

\usepackage{graphicx}
\usepackage{hyperref}
% MGMT 675 Beamer Style
% Include this file in your slide decks with: \input{mgmt675-style}

\usepackage{tcolorbox}
\tcbuselibrary{skins}
\usepackage{tikz}
\usepackage{booktabs}
\usetikzlibrary{shadings}

% Spacing adjustments
\linespread{0.9}  % Tighter baseline skip
\setlength{\parskip}{0.8\baselineskip plus 0.3ex}  % Extra space between paragraphs
\setbeamertemplate{itemize items}[circle]
\setlength{\leftmargini}{1.5em}

% Extra space between bullet items
\let\olditemize\itemize
\renewcommand{\itemize}{%
  \olditemize
  \setlength{\itemsep}{0.7ex}%
  \setlength{\parsep}{0.3ex}%
}

% Define accent color (matches title underline)
\definecolor{accentblue}{RGB}{37,99,235}
% Light blue/teal for shaded boxes
\definecolor{excelinput}{RGB}{232, 244, 247} %{177,214,224}
% Dark teal for titles
\definecolor{titlegray}{RGB}{19,60,71}

% Title slide highlight line color
\definecolor{titlehighlight}{RGB}{9,112,140}
\setbeamercolor{progress bar}{fg=titlehighlight}

% Alert text styling - bold bright coral/orange for contrast
\definecolor{alertorange}{RGB}{230,90,50}
\setbeamercolor{alerted text}{fg=alertorange}
\setbeamerfont{alerted text}{series=\bfseries}

% Hyperlink colors
\hypersetup{
  colorlinks=true,
  linkcolor=titlegray,
  urlcolor=alertorange,
  citecolor=accentblue
}

% Set frame title background
\setbeamercolor{frametitle}{bg=titlegray, fg=white}

% Style block environments with background shading
\setbeamercolor{block title}{fg=white, bg=accentblue}
\setbeamercolor{block body}{fg=black, bg=gray!10}

\setbeamercolor{block title alerted}{fg=white, bg=red!70!black}
\setbeamercolor{block body alerted}{fg=black, bg=red!10}

\setbeamercolor{block title example}{fg=white, bg=green!50!black}
\setbeamercolor{block body example}{fg=black, bg=green!10}

% Theorem environments
\setbeamercolor{theorem text}{fg=black}
\setbeamercolor{theoremblock title}{fg=white, bg=accentblue!80}
\setbeamercolor{theoremblock body}{fg=black, bg=accentblue!10}

% Bulleted list in shaded box
\newenvironment{baritemize}{%
  \begin{tcolorbox}[
    colback=excelinput,
    colframe=accentblue,
    boxrule=0.5pt,
    arc=2mm,
    auto outer arc,
    left=3mm,
    right=3mm,
    top=2mm,
    bottom=2mm
  ]
  \begin{itemize}
}{%
  \end{itemize}
  \end{tcolorbox}
}

% Numbered list in shaded box
\newenvironment{barenumerate}{%
  \begin{tcolorbox}[
    colback=excelinput,
    colframe=accentblue,
    boxrule=0.5pt,
    arc=2mm,
    auto outer arc,
    left=3mm,
    right=3mm,
    top=2mm,
    bottom=2mm
  ]
  \begin{enumerate}
}{%
  \end{enumerate}
  \end{tcolorbox}
}

% Define shaded box environment with slightly rounded corners and dark title bar
\newtcolorbox{shadedbox}[1][]{
  colback=excelinput,
  colframe=accentblue,
  boxrule=0.5pt,
  arc=2mm,
  auto outer arc,
  left=3mm,
  right=3mm,
  top=2mm,
  bottom=2mm,
  after skip=\baselineskip,
  coltitle=white,
  colbacktitle=titlegray,
  fonttitle=\bfseries,
  toptitle=2mm,
  bottomtitle=2mm,
  #1
}

% Gradient background for title slide
\let\oldmaketitle\maketitle
\renewcommand{\maketitle}{%
  {%
    \setbeamertemplate{background}{%
      \begin{tikzpicture}[remember picture,overlay]
        \fill[white] (current page.north west) rectangle (current page.south east);
        \shade[inner color=white, outer color=titlegray!40, shading=radial]
          (current page.north west) circle (28cm);
      \end{tikzpicture}%
    }%
    \oldmaketitle
  }%
}


\title{BUSI 722}
\subtitle{Session 3: Combining Signals \& Backtesting}
\author{Kerry Back}
\institute{}
\date{}

\begin{document}

\maketitle

%%% PART 1: COMBINING MULTIPLE SIGNALS %%%

\begin{frame}[c]
\centering
\Huge Combining Multiple Signals
\end{frame}

\begin{frame}{Three Methods}
We have multiple characteristics (momentum, bm, gpa, roe, \ldots).  How do we combine them to form portfolios?
\begin{enumerate}
\item \textbf{Intersecting sorts:} sort stocks into groups on each characteristic simultaneously and compare group returns.
\item \textbf{Composite ranks:} rank each stock 0 to 1 on each characteristic (1 $=$ best), then average across characteristics.  Sort on the composite.
\item \textbf{Fama-MacBeth prediction:} compute FM regression coefficients over a training period, average them, and apply the average coefficients to current characteristics to get a predicted value.  Sort on the prediction.
\end{enumerate}
\end{frame}

\begin{frame}{Exercise: Intersecting Sorts}
\begin{itemize}
\item Use a simple train-test split: data through 2019 for training, 2020 onward for testing.
\item Tell Claude to sort test-period stocks into quintiles on momentum and quintiles on book-to-market each month (25 groups). Compute the average return of each group each month.
\item Which corner of the 5$\times$5 table does best?  Which does worst?  Limitation: with more than 2--3 characteristics, groups become too sparse for reliable results.
\end{itemize}
\end{frame}

\begin{frame}{Exercise: Composite Ranks}
\begin{itemize}
\item Using the same train-test split, tell Claude to rank each stock from 0 to 1 each month on momentum, bm, gpa, and roe (ascending). Average the four ranks to get a composite score.
\item Sort test-period stocks into deciles on the composite each month and compute average decile returns.
\item Compare the top-bottom spread to what you got from intersecting sorts and single-characteristic sorts.
\end{itemize}
\end{frame}

\begin{frame}{Exercise: Fama-MacBeth Prediction}
\begin{itemize}
\item Using the same train-test split, tell Claude to run cross-sectional regressions of returns on momentum, bm, gpa, and roe each training month and average the coefficients.
\item Apply the average coefficients to test-period characteristics to get a predicted value for each stock each month.
\item Sort test-period stocks into deciles on the predicted value and compute average decile returns. Compare to composite ranks and intersecting sorts.
\end{itemize}
\end{frame}

\begin{frame}{Discussion}
\begin{itemize}
\item Which method produced the best spread?  The best Sharpe ratio?  Composite ranks and FM prediction both scale easily to many characteristics; intersecting sorts do not.
\item FM prediction is a \alert{linear} function of characteristics.  It accounts for correlations among characteristics but cannot capture nonlinear interactions.
\item Can we do better with a nonlinear model?  This motivates \alert{machine learning}.
\end{itemize}
\end{frame}

%%% PART 2: THE BACKTESTING PROCESS %%%

\begin{frame}[c]
\centering
\Huge The Backtesting Process
\end{frame}

\begin{frame}{Why Not $R^2$?}
\begin{itemize}
\item The conventional $R^2$ measures how well we predict the \alert{level} of returns, but to make money we only need to know which stocks will do \alert{relatively} well and which will do poorly.
\item Forecasting returns is extremely hard---monthly out-of-sample $R^2$ values of 0.5--1\% are considered excellent---yet even very low $R^2$ values can generate substantial portfolio profits.
\item The right evaluation metric is \alert{portfolio performance}, not $R^2$.
\end{itemize}
\end{frame}

\begin{frame}{Choosing the Target Variable}
What should we predict?
\begin{itemize}
\item \textbf{Raw returns:} simplest, but dominated by a common market component that is very hard to forecast.
\item \textbf{Returns minus the market return:} removes the hard-to-predict common risk and focuses the model on cross-sectional differences.
\item \textbf{Ranks or quantiles of returns:} we only need to get the ordering right. Rank and z-score each month, or classify into quintiles/deciles to turn the problem into classification.
\end{itemize}

\vspace{0.3cm}

In each case, we don't need to forecast returns precisely --- we just need to \alert{identify winners and losers}.
\end{frame}

\begin{frame}{Why Not Cross-Validation?}
\begin{itemize}
\item With cross-sectional data, we randomly split into train and test sets, or use $k$-fold cross-validation.
\item With time series, \alert{random splits use future data to predict the past}. Stock returns exhibit regime changes, trending volatility, and evolving factor premia, so a model trained on 2020 data and tested on 2015 data would have an unfair advantage.
\item We must \alert{always train on the past and test on the future}.
\end{itemize}
\end{frame}

\begin{frame}{Simple Train-Test Split}
\begin{itemize}
\item Train on data through date $T$, test on data after $T$. Example: train through 2019, test 2020--2025.
\item All test-period predictions are genuinely \alert{out of sample}.
\item This is the simplest valid backtesting setup --- good for experimenting with targets and portfolio formation before adding complexity.
\end{itemize}

\vspace{0.3cm}

Tell Claude to:
\begin{itemize}
\item Read the data and split into train (through 2019) and test (2020 onward).
\item Fit a model on the training data to predict next month's return from the features.
\item Use the fitted model to predict in the test period.
\end{itemize}
\end{frame}

\begin{frame}{Exercise: Comparing Targets}
Using the same features and simple train-test split, tell Claude to fit separate models predicting each of the following targets:
\begin{enumerate}
\item Raw returns
\item Returns minus the market return
\item Ranks of returns (z-scored each month) or quantile classification (e.g., top/bottom quintile)
\end{enumerate}

\vspace{0.3cm}

For each, sort test-period stocks into deciles based on predictions and compute average decile returns each month.  Which target produces the best \alert{spread} between the top and bottom deciles?
\end{frame}

\begin{frame}{Exercise: Comparing Portfolio Formation}
Using the best target from the previous exercise, compare different ways of forming portfolios from predictions:
\begin{enumerate}
\item Sort into deciles, go \textbf{long top decile} (long only).
\item Sort into deciles, go \textbf{long top, short bottom} (long-short).
\item Use predictions directly as \textbf{portfolio weights} (long-short, rescaled).
\end{enumerate}

\vspace{0.3cm}

For each, compute the mean monthly return, standard deviation, and Sharpe ratio over the test period.  Which approach performs best?
\end{frame}

\begin{frame}{From Static to Dynamic: Walk-Forward Validation}
\textbf{Problem with a simple split:}
\begin{itemize}
\item The model is static --- trained once on data through $T$ and never updated.
\item Markets change.  A model trained through 2019 may not work well in 2024.
\end{itemize}

\vspace{0.3cm}

\textbf{Walk-forward (rolling window) validation:}
\begin{enumerate}
\item Train on months 1 through $T$.  Predict month $T+1$.
\item Train on months 1 through $T+1$.  Predict month $T+2$.
\item Continue, always training on all available past data.
\end{enumerate}

\vspace{0.3cm}

Each prediction is genuinely \alert{out of sample}, and the model is \alert{continuously updated} with new information.
\end{frame}

\begin{frame}{The Backtesting Loop}
Each period, repeat the following steps:
\begin{enumerate}
\item \textbf{Train:} choose model and hyperparameters using past data only, then fit on the training window.
\item \textbf{Predict:} generate predictions for next period's returns.
\item \textbf{Form portfolio:} sort or weight stocks based on predictions.
\end{enumerate}

\vspace{0.3cm}

Then advance one period:
\begin{itemize}
\item Calculate the portfolio return over the period.
\item Add the new data to the training set.
\item Return to step 1 and repeat.
\end{itemize}
\end{frame}

\begin{frame}{Evaluating the Backtest}
\begin{itemize}
\item Sort stocks into deciles each month based on predicted values and compute the average return of each decile each month.
\item Evaluate the 10 out-of-sample portfolio return series: mean return, standard deviation, Sharpe ratio, cumulative performance.
\item The spread between the top and bottom deciles measures the \alert{predictive power} of the model.
\end{itemize}
\end{frame}

\begin{frame}{Training Pipeline}
\textbf{Three Main Steps:}
\begin{enumerate}
\item Create percentile-ranked features and train LightGBM with 12-month rolling windows
\item Form decile portfolios and analyze
\item Predict current month and save model
\end{enumerate}

\vspace{0.3cm}

\textbf{Key Parameters:}
\begin{itemize}
\item \texttt{TRAINING\_WINDOW = 12} months
\item \texttt{N\_PORTFOLIOS = 10} deciles
\item LightGBM with 100 trees, learning rate 0.05, max depth 6
\end{itemize}
\end{frame}

\begin{frame}{Portfolio Results}
\begin{itemize}
\item Average monthly spread (D10 $-$ D1): 2.50\%
\item Decile portfolio returns from sorting on LightGBM predictions
\end{itemize}
\vspace{0.3cm}
\textbf{Current Month Predictions:}
\begin{itemize}
\item Trains on last 12 complete months
\item Outputs predictions sorted highest to lowest
\item Includes all features for current month analysis
\end{itemize}
\end{frame}

\begin{frame}{Portfolio Analysis Notebook}
\begin{enumerate}
\item Mean returns and Sharpe ratios bar charts by decile
\item Cumulative returns (linear and log scale) and summary statistics (mean, volatility, Sharpe, min, max)
\item Long-short portfolio -- D10 $-$ D1 spread performance
\end{enumerate}
\end{frame}

\begin{frame}{Model Feature Analysis}
\begin{enumerate}
\item Feature importances pie chart -- from LightGBM split gain
\item Linear regression of predictions on percentile-ranked features with coefficient bar chart
\item Comparison table -- feature importance ranks vs.\ coefficient ranks (large differences reveal non-linear effects)
\end{enumerate}
\end{frame}

\begin{frame}{Exercise: Run and Analyze a Backtest}
\begin{enumerate}
\item Tell Claude to run the LightGBM rolling-window pipeline on the dataset.
\item Create a Jupyter notebook that plots mean returns by decile, cumulative returns for top and bottom deciles, and the long-short spread.
\item Which deciles perform best and worst?  Is the spread monotonic?  Examine feature importances for surprises.
\end{enumerate}
\end{frame}

\end{document}
