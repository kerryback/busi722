\documentclass[aspectratio=169]{beamer}
\usetheme{metropolis}

\usepackage{graphicx}
\usepackage{hyperref}
% MGMT 675 Beamer Style
% Include this file in your slide decks with: \input{mgmt675-style}

\usepackage{tcolorbox}
\tcbuselibrary{skins}
\usepackage{tikz}
\usepackage{booktabs}
\usetikzlibrary{shadings}

% Spacing adjustments
\linespread{0.9}  % Tighter baseline skip
\setlength{\parskip}{0.8\baselineskip plus 0.3ex}  % Extra space between paragraphs
\setbeamertemplate{itemize items}[circle]
\setlength{\leftmargini}{1.5em}

% Extra space between bullet items
\let\olditemize\itemize
\renewcommand{\itemize}{%
  \olditemize
  \setlength{\itemsep}{0.7ex}%
  \setlength{\parsep}{0.3ex}%
}

% Define accent color (matches title underline)
\definecolor{accentblue}{RGB}{37,99,235}
% Light blue/teal for shaded boxes
\definecolor{excelinput}{RGB}{232, 244, 247} %{177,214,224}
% Dark teal for titles
\definecolor{titlegray}{RGB}{19,60,71}

% Title slide highlight line color
\definecolor{titlehighlight}{RGB}{9,112,140}
\setbeamercolor{progress bar}{fg=titlehighlight}

% Alert text styling - bold bright coral/orange for contrast
\definecolor{alertorange}{RGB}{230,90,50}
\setbeamercolor{alerted text}{fg=alertorange}
\setbeamerfont{alerted text}{series=\bfseries}

% Hyperlink colors
\hypersetup{
  colorlinks=true,
  linkcolor=titlegray,
  urlcolor=alertorange,
  citecolor=accentblue
}

% Set frame title background
\setbeamercolor{frametitle}{bg=titlegray, fg=white}

% Style block environments with background shading
\setbeamercolor{block title}{fg=white, bg=accentblue}
\setbeamercolor{block body}{fg=black, bg=gray!10}

\setbeamercolor{block title alerted}{fg=white, bg=red!70!black}
\setbeamercolor{block body alerted}{fg=black, bg=red!10}

\setbeamercolor{block title example}{fg=white, bg=green!50!black}
\setbeamercolor{block body example}{fg=black, bg=green!10}

% Theorem environments
\setbeamercolor{theorem text}{fg=black}
\setbeamercolor{theoremblock title}{fg=white, bg=accentblue!80}
\setbeamercolor{theoremblock body}{fg=black, bg=accentblue!10}

% Bulleted list in shaded box
\newenvironment{baritemize}{%
  \begin{tcolorbox}[
    colback=excelinput,
    colframe=accentblue,
    boxrule=0.5pt,
    arc=2mm,
    auto outer arc,
    left=3mm,
    right=3mm,
    top=2mm,
    bottom=2mm
  ]
  \begin{itemize}
}{%
  \end{itemize}
  \end{tcolorbox}
}

% Numbered list in shaded box
\newenvironment{barenumerate}{%
  \begin{tcolorbox}[
    colback=excelinput,
    colframe=accentblue,
    boxrule=0.5pt,
    arc=2mm,
    auto outer arc,
    left=3mm,
    right=3mm,
    top=2mm,
    bottom=2mm
  ]
  \begin{enumerate}
}{%
  \end{enumerate}
  \end{tcolorbox}
}

% Define shaded box environment with slightly rounded corners and dark title bar
\newtcolorbox{shadedbox}[1][]{
  colback=excelinput,
  colframe=accentblue,
  boxrule=0.5pt,
  arc=2mm,
  auto outer arc,
  left=3mm,
  right=3mm,
  top=2mm,
  bottom=2mm,
  after skip=\baselineskip,
  coltitle=white,
  colbacktitle=titlegray,
  fonttitle=\bfseries,
  toptitle=2mm,
  bottomtitle=2mm,
  #1
}

% Gradient background for title slide
\let\oldmaketitle\maketitle
\renewcommand{\maketitle}{%
  {%
    \setbeamertemplate{background}{%
      \begin{tikzpicture}[remember picture,overlay]
        \fill[white] (current page.north west) rectangle (current page.south east);
        \shade[inner color=white, outer color=titlegray!40, shading=radial]
          (current page.north west) circle (28cm);
      \end{tikzpicture}%
    }%
    \oldmaketitle
  }%
}


\title{BUSI 722}
\subtitle{Session 7: Direct Portfolio Construction}
\author{Kerry Back}
\institute{}
\date{}

\begin{document}

\maketitle

%%% PART 1: DIRECT PORTFOLIO CONSTRUCTION %%%

\begin{frame}[c]
\centering
\Huge Direct Portfolio Construction from Signals
\end{frame}

\begin{frame}{The Problem with Predict-Then-Optimize}
\textbf{Traditional approach:}
\begin{barenumerate}
\item Estimate expected returns $\hat\mu$ and covariance matrix $\hat\Sigma$
\item Feed $\hat\mu$ and $\hat\Sigma$ into a portfolio optimizer
\end{barenumerate}

\vspace{0.3cm}

\textbf{Why this fails in practice:}
\begin{itemize}
\item DeMiguel, Garlappi, Uppal (\textit{RFS}, 2009): none of 14 optimized portfolios consistently beats equal weighting out of sample
\item Need ${\sim}$3,000 months of data for mean-variance to beat $1/N$ with 25 assets
\item Estimation error in $\hat\mu$ and $\hat\Sigma$ is \alert{amplified} by the optimizer
\end{itemize}

\vspace{0.3cm}

\textbf{Alternative:} bypass moment estimation entirely and map signals directly to portfolio weights.
\end{frame}

\begin{frame}{Parametric Portfolio Policies}
\textbf{Brandt, Santa-Clara, Valkanov} (\textit{RFS}, 2009)

\vspace{0.2cm}

Portfolio weights are a \alert{direct function of stock characteristics}:
$$w_{i,t} = \bar w_{i,t} + \frac{1}{N_t}\,\theta'\hat x_{i,t}$$
\begin{itemize}
\item $\bar w_{i,t}$ = benchmark weight (e.g., value-weighted market)
\item $\hat x_{i,t}$ = cross-sectionally standardized characteristics (zero mean, unit std each month)
\item $\theta$ = policy parameters estimated by maximizing expected CRRA utility
\end{itemize}

\vspace{0.3cm}

\textbf{Key insight:} estimate $K$ parameters (one per characteristic) instead of $N$ means $+$ $N(N\!+\!1)/2$ covariances.  With $K=3$ (size, value, momentum), this is tractable for any number of stocks.

\vspace{0.2cm}

\textbf{Result:} ${\sim}$10\% annualized certainty-equivalent gain over the market portfolio.
\end{frame}

\begin{frame}{Principal Portfolios}
\textbf{Kelly, Malamud, Pedersen} (\textit{J.\ Finance}, 2023)

\vspace{0.2cm}

\begin{itemize}
\item Define a \textbf{prediction matrix} $M$ where $M_{ij}$ captures how signal $j$ predicts return $i$ --- including \alert{cross-asset predictability}.
\item The optimal portfolio is the \textbf{eigenvector} of $M$ (by analogy to PCA on the covariance matrix).
\item Decompose $M$ into symmetric and antisymmetric parts:
\begin{itemize}
\item Symmetric $\Rightarrow$ \textbf{beta portfolios} (factor risk premia)
\item Antisymmetric $\Rightarrow$ \textbf{alpha portfolios} (market-neutral)
\end{itemize}
\item Provides the \alert{theoretical foundation} for why characteristic-based portfolios work.
\item Brandt et al.\ is a special case (diagonal $M$, linear policy).
\end{itemize}
\end{frame}

\begin{frame}{The Virtue of Complexity}
\textbf{Kelly, Malamud, Zhou} (\textit{J.\ Finance}, 2024)

\vspace{0.3cm}

\begin{itemize}
\item Proves that \textbf{more complex models generically outperform simpler ones} for return prediction and portfolio construction.
\item Out-of-sample $R^2$ and optimal Sharpe ratio \alert{increase with model parameterization}, even when parameters exceed observations.
\item Holds in extremely data-scarce environments ($<20$ observations, tens of thousands of predictors).
\item ML strategies learn to divest before recessions --- successful in 14 of 15 NBER recessions out of sample.
\end{itemize}

\vspace{0.3cm}

\textbf{Practical implication:} use the largest model you can compute.
\end{frame}

\begin{frame}{Random Fourier Features + Ridge Regression}
\textbf{Didisheim, Ke, Kelly, Malamud} (2024)

\vspace{0.2cm}

\textbf{The idea:} convert $d$ raw characteristics $X_t$ into $P$ nonlinear features via random projections:
$$S_{i,t} = \bigl[\sin(\gamma\, X_t\omega_i),\;\cos(\gamma\, X_t\omega_i)\bigr]', \quad \omega_i \sim \text{i.i.d.}\ \mathcal{N}(0,I)$$
\begin{itemize}
\item Equivalent to a wide two-layer neural network with \alert{fixed random weights} in the first layer
\item Only the second layer (portfolio weights $\lambda$) is estimated --- by \textbf{ridge regression}
\item Can scale $P$ from 2 to 1,000,000 from the same raw data
\end{itemize}
\end{frame}

\begin{frame}{RFF $\approx$ Neural Network with Frozen First Layer}
\begin{center}
\includegraphics[width=\textwidth]{rff_vs_nn.pdf}
\end{center}

\vspace{-0.2cm}

\small
\begin{columns}[T]
\begin{column}{0.48\textwidth}
\textbf{Neural network:} both layers trained jointly via backpropagation.  Powerful but expensive and prone to local minima.
\end{column}
\begin{column}{0.48\textwidth}
\textbf{Random Fourier features:} first-layer weights drawn randomly and \alert{never updated}.  Second layer solved in \alert{closed form} via ridge regression.
\end{column}
\end{columns}
\end{frame}

\begin{frame}{Random Fourier Features: Results}
\textbf{Key findings} (Didisheim et al., 2024):
\begin{itemize}
\item Out-of-sample SDF Sharpe ratio rises from ${\sim}1$ (low complexity) to ${\sim}4$ (high complexity) --- far exceeding Fama-French 6-factor model (SR $\approx 1.1$)
\item Out-of-sample pricing errors fall by a factor of six as $P$ grows
\item Even using only the 5 Fama-French characteristics, adding nonlinear complexity \alert{doubles the Sharpe ratio} and cuts pricing errors by half
\item Robust across market cap groups (mega, large, small, micro)
\end{itemize}

\vspace{0.3cm}

\textbf{Why it matters:} achieves performance rivaling deep learning with \alert{no backpropagation} --- just random projections + ridge regression.
\end{frame}

\begin{frame}{Exercise: Parametric Portfolio Policies}
\begin{barenumerate}
\item Tell Claude to implement Brandt et al.\ with three characteristics: momentum, book-to-market, and profitability (gpa).
\item Cross-sectionally standardize each characteristic each month (zero mean, unit standard deviation).
\item Estimate $\theta$ by maximizing average portfolio return minus $\frac{\gamma}{2}$ times portfolio variance, with $\gamma = 5$.
\item Compare the Sharpe ratio to the equal-weighted portfolio and to the LightGBM decile portfolio from Session 3.
\end{barenumerate}
\end{frame}

%%% PART 2: DEEP LEARNING EXTENSIONS %%%

\begin{frame}[c]
\centering
\Huge Deep Learning Extensions
\end{frame}

\begin{frame}{Deep Parametric Portfolio Policies}

\textbf{Simon, Weibels, Zimmermann} (2023):
\begin{itemize}
\item Replace Brandt et al.'s linear weight function with a \textbf{feed-forward neural network}
\item Captures nonlinear interactions among characteristics
\item 75--276 bps/month improvement in certainty equivalent returns
\item Risk aversion $\gamma$ serves as an economically motivated regularization parameter
\end{itemize}

\vspace{0.3cm}

\textbf{AI Asset Pricing Models} (Kelly, Kuznetsov, Malamud, Xu, 2025):
\begin{itemize}
\item Embeds a \textbf{transformer} in the stochastic discount factor
\item \textbf{Cross-asset attention} captures interactions across the full stock universe
\item Large reductions in pricing errors vs.\ previous ML models
\item Current frontier of the field
\end{itemize}
\end{frame}

\begin{frame}{References: Direct Portfolio Construction}
\small
\begin{itemize}
\item DeMiguel, Garlappi, Uppal (2009), ``Optimal Versus Naive Diversification,'' \textit{RFS}
\item Brandt, Santa-Clara, Valkanov (2009), ``Parametric Portfolio Policies,'' \textit{RFS}
\item Kelly, Malamud, Pedersen (2023), ``Principal Portfolios,'' \textit{J.\ Finance}
\item Kelly, Malamud, Zhou (2024), ``The Virtue of Complexity in Return Prediction,'' \textit{J.\ Finance}
\item Didisheim, Ke, Kelly, Malamud (2024), ``Complexity in Factor Pricing Models,'' Working Paper
\item Simon, Weibels, Zimmermann (2023), ``Deep Parametric Portfolio Policies,'' CFR Working Paper
\item Kelly, Kuznetsov, Malamud, Xu (2025), ``Artificial Intelligence Asset Pricing Models,'' NBER WP 33351
\item Gu, Kelly, Xiu (2020), ``Empirical Asset Pricing via Machine Learning,'' \textit{RFS}
\end{itemize}
\end{frame}

%%% PART 3: WRAP-UP %%%

\begin{frame}{Using Apps and Scripts with AI}
\begin{itemize}
\item Where possible, permanently fix the code used for tasks rather than asking AI to regenerate it each time.
\item This ensures consistent behavior and reduces token usage.
\item Save as a .py file and use in a Claude Code skill.
\end{itemize}
\end{frame}

\end{document}
