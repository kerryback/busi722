\documentclass[aspectratio=169]{beamer}
\usetheme{metropolis}

\usepackage{graphicx}
\usepackage{hyperref}
% MGMT 675 Beamer Style
% Include this file in your slide decks with: \input{mgmt675-style}

\usepackage{tcolorbox}
\tcbuselibrary{skins}
\usepackage{tikz}
\usepackage{booktabs}
\usetikzlibrary{shadings}

% Spacing adjustments
\linespread{0.9}  % Tighter baseline skip
\setlength{\parskip}{0.8\baselineskip plus 0.3ex}  % Extra space between paragraphs
\setbeamertemplate{itemize items}[circle]
\setlength{\leftmargini}{1.5em}

% Extra space between bullet items
\let\olditemize\itemize
\renewcommand{\itemize}{%
  \olditemize
  \setlength{\itemsep}{0.7ex}%
  \setlength{\parsep}{0.3ex}%
}

% Define accent color (matches title underline)
\definecolor{accentblue}{RGB}{37,99,235}
% Light blue/teal for shaded boxes
\definecolor{excelinput}{RGB}{232, 244, 247} %{177,214,224}
% Dark teal for titles
\definecolor{titlegray}{RGB}{19,60,71}

% Title slide highlight line color
\definecolor{titlehighlight}{RGB}{9,112,140}
\setbeamercolor{progress bar}{fg=titlehighlight}

% Alert text styling - bold bright coral/orange for contrast
\definecolor{alertorange}{RGB}{230,90,50}
\setbeamercolor{alerted text}{fg=alertorange}
\setbeamerfont{alerted text}{series=\bfseries}

% Hyperlink colors
\hypersetup{
  colorlinks=true,
  linkcolor=titlegray,
  urlcolor=alertorange,
  citecolor=accentblue
}

% Set frame title background
\setbeamercolor{frametitle}{bg=titlegray, fg=white}

% Style block environments with background shading
\setbeamercolor{block title}{fg=white, bg=accentblue}
\setbeamercolor{block body}{fg=black, bg=gray!10}

\setbeamercolor{block title alerted}{fg=white, bg=red!70!black}
\setbeamercolor{block body alerted}{fg=black, bg=red!10}

\setbeamercolor{block title example}{fg=white, bg=green!50!black}
\setbeamercolor{block body example}{fg=black, bg=green!10}

% Theorem environments
\setbeamercolor{theorem text}{fg=black}
\setbeamercolor{theoremblock title}{fg=white, bg=accentblue!80}
\setbeamercolor{theoremblock body}{fg=black, bg=accentblue!10}

% Bulleted list in shaded box
\newenvironment{baritemize}{%
  \begin{tcolorbox}[
    colback=excelinput,
    colframe=accentblue,
    boxrule=0.5pt,
    arc=2mm,
    auto outer arc,
    left=3mm,
    right=3mm,
    top=2mm,
    bottom=2mm
  ]
  \begin{itemize}
}{%
  \end{itemize}
  \end{tcolorbox}
}

% Numbered list in shaded box
\newenvironment{barenumerate}{%
  \begin{tcolorbox}[
    colback=excelinput,
    colframe=accentblue,
    boxrule=0.5pt,
    arc=2mm,
    auto outer arc,
    left=3mm,
    right=3mm,
    top=2mm,
    bottom=2mm
  ]
  \begin{enumerate}
}{%
  \end{enumerate}
  \end{tcolorbox}
}

% Define shaded box environment with slightly rounded corners and dark title bar
\newtcolorbox{shadedbox}[1][]{
  colback=excelinput,
  colframe=accentblue,
  boxrule=0.5pt,
  arc=2mm,
  auto outer arc,
  left=3mm,
  right=3mm,
  top=2mm,
  bottom=2mm,
  after skip=\baselineskip,
  coltitle=white,
  colbacktitle=titlegray,
  fonttitle=\bfseries,
  toptitle=2mm,
  bottomtitle=2mm,
  #1
}

% Gradient background for title slide
\let\oldmaketitle\maketitle
\renewcommand{\maketitle}{%
  {%
    \setbeamertemplate{background}{%
      \begin{tikzpicture}[remember picture,overlay]
        \fill[white] (current page.north west) rectangle (current page.south east);
        \shade[inner color=white, outer color=titlegray!40, shading=radial]
          (current page.north west) circle (28cm);
      \end{tikzpicture}%
    }%
    \oldmaketitle
  }%
}


\title{BUSI 722}
\subtitle{Pre-Session Reading: Quantitative Signals for Stock Selection}
\author{Kerry Back}
\institute{}
\date{}

\begin{document}

\maketitle

\begin{frame}{The Goal}
\begin{itemize}
\item Use data to generate \alert{quantitative signals} that predict which stocks will outperform.
\item Combine multiple signals using machine learning.
\item Build portfolios, backtest, and evaluate performance.
\item This course: use AI (Claude Code) to do all of the above.
\end{itemize}
\end{frame}

%%% PART 1A: PREDICTORS FROM PAST RETURNS %%%

\begin{frame}[c]
\centering
\Huge Predictors from Past Returns
\end{frame}

\begin{frame}{Momentum}
\begin{itemize}
\item Jegadeesh \& Titman (1993): buy stocks with high past 3--12 month returns, sell stocks with low past returns.
\item Skip the most recent month (short-term reversal contaminates the signal).
\item One of the most robust anomalies in finance---documented across countries, asset classes, and time periods.
\item Typical implementation: sort on cumulative return from $t{-}12$ to $t{-}2$.
\end{itemize}
\end{frame}

\begin{frame}{Time-Series Momentum}
\begin{itemize}
\item Moskowitz, Ooi \& Pedersen (2012): go long assets with positive past 12-month returns, short those with negative past returns.
\item Differs from cross-sectional momentum: each asset is compared to \alert{its own past}, not to other assets.
\item Works across equities, bonds, commodities, and currencies.
\item Related to trend-following strategies used by managed futures funds.
\end{itemize}
\end{frame}

\begin{frame}{Moving Averages}
\begin{itemize}
\item SMA (simple moving average) and EMA (exponential moving average) smooth price series over a window.
\item \textbf{200-day rule} (Faber 2007): buy when price is above its 200-day MA, sell when below.
\item Brock, Lakonishok \& LeBaron (1992) found strong support for MA trading rules in the DJIA.
\item Zhu \& Zhou (2009) provided theoretical justification: MAs optimally combine momentum signals at different horizons.
\end{itemize}
\end{frame}

\begin{frame}{Trend Factor}
\begin{itemize}
\item Han, Zhou \& Zhu (2016): construct a single factor from moving averages at multiple horizons.
\item Combines short-, medium-, and long-term trend information into one signal.
\item Earns large risk-adjusted returns that are not explained by standard factors.
\item Performs well during crises when standard momentum crashes.
\end{itemize}
\end{frame}

\begin{frame}{Reversals}
\begin{itemize}
\item \textbf{Short-term reversal:} last month's losers outperform next month (Lehmann 1990, Jegadeesh 1990). Driven by liquidity provision and overreaction.
\item \textbf{Long-term reversal:} 3--5 year past winners underperform (DeBondt \& Thaler 1985). Consistent with investor overreaction and mean reversion.
\item Together with momentum, these define the \alert{complete horizon structure} of return predictability:
\begin{itemize}
\item 1 month: reversal
\item 2--12 months: momentum
\item 3--5 years: reversal
\end{itemize}
\end{itemize}
\end{frame}

\begin{frame}{52-Week High}
\begin{itemize}
\item George \& Hwang (2004): nearness to 52-week high predicts future returns.
\item Signal: current price / 52-week high price.
\item Outperforms standard momentum as a predictor and profits do not reverse.
\item Behavioral explanation: \alert{anchoring bias}---investors use the 52-week high as a reference point and underreact when price approaches it.
\end{itemize}
\end{frame}

\begin{frame}{Other Technical Indicators}
\begin{itemize}
\item \textbf{RSI} (Relative Strength Index): overbought above 70, oversold below 30. Mean-reversion signal.
\item \textbf{MACD} (Moving Average Convergence Divergence): difference between short and long EMAs; signal line crossovers indicate trend changes.
\item \textbf{Bollinger Bands:} price within a volatility envelope ($\pm 2$ standard deviations around 20-day MA). Profitability largely disappeared post-publication.
\item Widely used by practitioners, but limited top-journal academic support for out-of-sample profitability.
\end{itemize}
\end{frame}

\begin{frame}{Volatility}
\begin{itemize}
\item \textbf{Low-volatility anomaly:} stocks with low historical volatility earn higher \alert{risk-adjusted} returns than high-volatility stocks.
\item Contradicts the basic risk--return tradeoff predicted by CAPM.
\item Explanations: leverage constraints, lottery preferences, benchmarking by institutional investors.
\item Ang, Hodrick, Xing \& Zhang (2006) documented the effect for idiosyncratic volatility.
\end{itemize}
\end{frame}

%%% PART 1B: PREDICTORS FROM FINANCIAL STATEMENTS %%%

\begin{frame}[c]
\centering
\Huge Predictors from Financial Statements
\end{frame}

\begin{frame}{Value}
\begin{itemize}
\item \textbf{Book-to-market:} Fama \& French (1992) documented that high B/M stocks (``value'') outperform low B/M stocks (``growth'').
\item \textbf{Earnings yield:} earnings / price. Basu (1977) showed high E/P stocks earn higher returns.
\item \textbf{Cash flow yield:} cash flow / price. Lakonishok, Shleifer \& Vishny (1994).
\item Common theme: stocks with high ratios of fundamental value to market price tend to outperform.
\end{itemize}
\end{frame}

\begin{frame}{Profitability}
\begin{itemize}
\item Novy-Marx (2013): gross profit / assets is a strong predictor of returns.
\item More profitable firms earn higher returns, especially when combined with value.
\item ROE, operating margins, and other profitability measures also predict.
\item Fama \& French (2015) added profitability and investment factors to their three-factor model.
\end{itemize}
\end{frame}

\begin{frame}{Growth \& Investment}
\begin{itemize}
\item \textbf{Asset growth:} Cooper, Gulen \& Schill (2008) showed firms that grow assets aggressively tend to \alert{underperform}.
\item \textbf{Investment / assets:} similar finding---high capital expenditure predicts lower returns.
\item This is the \alert{investment anomaly}: empire-building firms destroy shareholder value.
\item Consistent with overinvestment by managers or diminishing returns to capital.
\end{itemize}
\end{frame}

\begin{frame}{Quality \& Accruals}
\begin{itemize}
\item \textbf{Accruals anomaly} (Sloan 1996): firms with high accruals (earnings far above cash flow) underperform. Cash-based earnings are more persistent.
\item \textbf{Quality Minus Junk} (Asness, Frazzini \& Pedersen 2019): composite of profitability, growth, safety, and payout.
\item Other quality signals: earnings stability, low leverage, high payout ratios.
\item Quality stocks earn higher returns despite being ``safer''---another puzzle for standard asset pricing.
\end{itemize}
\end{frame}

%%% PART 1C: OTHER PREDICTORS %%%

\begin{frame}[c]
\centering
\Huge Other Predictors
\end{frame}

\begin{frame}{Short Interest}
\begin{itemize}
\item Heavily shorted stocks tend to underperform (Desai et al.\ 2002, Rapach, Ringgenberg \& Zhou 2016).
\item Short interest reflects \alert{informed pessimistic opinions}: short sellers face costs (borrowing fees, margin, unlimited loss potential), so they act only on strong convictions.
\item Aggregate short interest also forecasts market-level returns.
\item Constraints on short selling limit the speed of price adjustment to negative information.
\end{itemize}
\end{frame}

\begin{frame}{Insider Trades}
\begin{itemize}
\item Insider purchases predict positive future returns (Lakonishok \& Lee 2001, Jeng, Metrick \& Zeckhauser 2003).
\item Insider sales are less informative---insiders sell for many reasons (diversification, liquidity needs).
\item Insiders have an \alert{informational advantage} about their own firms' prospects.
\item SEC filings (Form 4) make insider trades public, typically within two business days.
\end{itemize}
\end{frame}

\begin{frame}{Earnings Announcements}
\begin{itemize}
\item \textbf{Post-earnings announcement drift (PEAD):} one of the oldest documented anomalies (Ball \& Brown 1968, Bernard \& Thomas 1989).
\item Prices drift in the direction of the earnings surprise for weeks or months after the announcement.
\item \textbf{Standardized unexpected earnings (SUE):} (actual $-$ expected) / standard deviation. Higher SUE predicts higher future returns.
\item Explanation: investors underreact to earnings news and update beliefs too slowly.
\end{itemize}
\end{frame}

\begin{frame}{Analyst Revisions}
\begin{itemize}
\item Changes in analyst earnings estimates and price targets predict returns.
\item Upward revisions are followed by positive returns; downward revisions by negative returns.
\item Information from analyst revisions is incorporated into prices slowly, similar to PEAD.
\item Analysts aggregate information from multiple sources, making revisions a useful summary signal.
\end{itemize}
\end{frame}

%%% WHAT WE WILL USE %%%

\begin{frame}{What We Will Use}
\begin{itemize}
\item This course focuses on signals from \textbf{past returns} and \textbf{financial statements}, available in the Rice Business stock market database.
\item We will construct signals like momentum, moving averages, book-to-market, profitability ratios, and more directly from the database.
\item The methods we develop (sorting, Fama-MacBeth regression, machine learning, backtesting) apply equally to any signal source.
\end{itemize}
\end{frame}

\end{document}
