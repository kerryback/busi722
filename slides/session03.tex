\documentclass[aspectratio=169]{beamer}
\usetheme{metropolis}

\usepackage{graphicx}
\usepackage{hyperref}
\usepackage{amsmath}
% MGMT 675 Beamer Style
% Include this file in your slide decks with: \input{mgmt675-style}

\usepackage{tcolorbox}
\tcbuselibrary{skins}
\usepackage{tikz}
\usepackage{booktabs}
\usetikzlibrary{shadings}

% Spacing adjustments
\linespread{0.9}  % Tighter baseline skip
\setlength{\parskip}{0.8\baselineskip plus 0.3ex}  % Extra space between paragraphs
\setbeamertemplate{itemize items}[circle]
\setlength{\leftmargini}{1.5em}

% Extra space between bullet items
\let\olditemize\itemize
\renewcommand{\itemize}{%
  \olditemize
  \setlength{\itemsep}{0.7ex}%
  \setlength{\parsep}{0.3ex}%
}

% Define accent color (matches title underline)
\definecolor{accentblue}{RGB}{37,99,235}
% Light blue/teal for shaded boxes
\definecolor{excelinput}{RGB}{232, 244, 247} %{177,214,224}
% Dark teal for titles
\definecolor{titlegray}{RGB}{19,60,71}

% Title slide highlight line color
\definecolor{titlehighlight}{RGB}{9,112,140}
\setbeamercolor{progress bar}{fg=titlehighlight}

% Alert text styling - bold bright coral/orange for contrast
\definecolor{alertorange}{RGB}{230,90,50}
\setbeamercolor{alerted text}{fg=alertorange}
\setbeamerfont{alerted text}{series=\bfseries}

% Hyperlink colors
\hypersetup{
  colorlinks=true,
  linkcolor=titlegray,
  urlcolor=alertorange,
  citecolor=accentblue
}

% Set frame title background
\setbeamercolor{frametitle}{bg=titlegray, fg=white}

% Style block environments with background shading
\setbeamercolor{block title}{fg=white, bg=accentblue}
\setbeamercolor{block body}{fg=black, bg=gray!10}

\setbeamercolor{block title alerted}{fg=white, bg=red!70!black}
\setbeamercolor{block body alerted}{fg=black, bg=red!10}

\setbeamercolor{block title example}{fg=white, bg=green!50!black}
\setbeamercolor{block body example}{fg=black, bg=green!10}

% Theorem environments
\setbeamercolor{theorem text}{fg=black}
\setbeamercolor{theoremblock title}{fg=white, bg=accentblue!80}
\setbeamercolor{theoremblock body}{fg=black, bg=accentblue!10}

% Bulleted list in shaded box
\newenvironment{baritemize}{%
  \begin{tcolorbox}[
    colback=excelinput,
    colframe=accentblue,
    boxrule=0.5pt,
    arc=2mm,
    auto outer arc,
    left=3mm,
    right=3mm,
    top=2mm,
    bottom=2mm
  ]
  \begin{itemize}
}{%
  \end{itemize}
  \end{tcolorbox}
}

% Numbered list in shaded box
\newenvironment{barenumerate}{%
  \begin{tcolorbox}[
    colback=excelinput,
    colframe=accentblue,
    boxrule=0.5pt,
    arc=2mm,
    auto outer arc,
    left=3mm,
    right=3mm,
    top=2mm,
    bottom=2mm
  ]
  \begin{enumerate}
}{%
  \end{enumerate}
  \end{tcolorbox}
}

% Define shaded box environment with slightly rounded corners and dark title bar
\newtcolorbox{shadedbox}[1][]{
  colback=excelinput,
  colframe=accentblue,
  boxrule=0.5pt,
  arc=2mm,
  auto outer arc,
  left=3mm,
  right=3mm,
  top=2mm,
  bottom=2mm,
  after skip=\baselineskip,
  coltitle=white,
  colbacktitle=titlegray,
  fonttitle=\bfseries,
  toptitle=2mm,
  bottomtitle=2mm,
  #1
}

% Gradient background for title slide
\let\oldmaketitle\maketitle
\renewcommand{\maketitle}{%
  {%
    \setbeamertemplate{background}{%
      \begin{tikzpicture}[remember picture,overlay]
        \fill[white] (current page.north west) rectangle (current page.south east);
        \shade[inner color=white, outer color=titlegray!40, shading=radial]
          (current page.north west) circle (28cm);
      \end{tikzpicture}%
    }%
    \oldmaketitle
  }%
}


\title{Machine Learning for Return Prediction}
\subtitle{BUSI 722: Data-Driven Finance II}
\author{Kerry Back}
\institute{}
\date{}

\begin{document}

\maketitle

%%% PART 1: STANDARDIZING FEATURES %%%

\begin{frame}[c]
\centering
\Huge Standardizing Features
\end{frame}

\begin{frame}{Why Standardize?}
\begin{itemize}
\item Raw features have very different scales: marketcap is in thousands, roe is a fraction, momentum can be $\pm 100\%$.
\item Many models (linear regression, neural networks) are sensitive to scale.
\item Standardizing puts all features on a common scale and tames outliers.
\end{itemize}
\end{frame}

\begin{frame}{Cross-Sectional Standardization}
Standardize each feature \alert{within each month}, not over the full panel.
\vspace{0.3cm}

For feature $x$ in month $t$:
\[
z_{i,t} = \frac{x_{i,t} - \bar x_t}{\sigma_{x,t}}
\]

\begin{itemize}
\item Each month's cross section has mean 0 and standard deviation 1.
\item Removes time-series level shifts (e.g., all P/E ratios rising over time).
\item Ensures the model learns \alert{which stocks are cheap relative to their peers this month}, not which months had low valuations overall.
\end{itemize}
\end{frame}

\begin{frame}{Ranks as an Alternative}
Instead of z-scores, rank stocks 1 through $n$ each month, then scale to $[0, 1]$:
\[
\text{rank}_{i,t} = \frac{\text{rank of } x_{i,t} \text{ among all stocks in month } t}{n_t}
\]

\begin{itemize}
\item Completely eliminates outliers --- every stock gets a value in $[0,1]$.
\item Preserves only the \alert{ordering} of stocks, not the magnitudes.
\item Robust to skewed distributions; can also compute z-scores of ranks for a hybrid approach.
\end{itemize}
\end{frame}

\begin{frame}{Which Features Are ``Good''?}
When using ranks, we want a \alert{high rank to always mean good}.
\vspace{0.3cm}

\begin{itemize}
\item \textbf{Higher is better:} momentum, roe, grossmargin $\Rightarrow$ rank low to high.
\item \textbf{Lower is better:} asset growth (conservatism effect) $\Rightarrow$ rank high to low.
\end{itemize}

\vspace{0.3cm}
Alternatively, just rank all features the same way and let the model learn the sign. This is simpler and works well with flexible models (trees, neural nets).
\end{frame}

%%% PART 2: STANDARDIZING THE TARGET %%%

\begin{frame}[c]
\centering
\Huge Standardizing the Target
\end{frame}

\begin{frame}{Predicting Relative Returns}
We care more about \alert{which stocks will outperform} than about predicting exact return magnitudes.
\vspace{0.3cm}

\alert{If every stock's return rises by 2\% because the market rallies, that tells us nothing about which stocks to buy.  What matters is the cross-sectional ranking of returns.}

\vspace{0.3cm}
This suggests we should think carefully about what target variable we give the model.
\end{frame}

\begin{frame}{Ranks as the Target}
Replace the raw return $r_{i,t}$ with its \alert{cross-sectional rank} each month:
\[
y_{i,t} = \frac{\text{rank of } r_{i,t} \text{ among all stocks in month } t}{n_t}
\]

\begin{itemize}
\item The model learns to predict \alert{relative performance}, not absolute returns.
\item Removes the influence of market-wide shocks and reduces the impact of return outliers.
\item Scaled to $[0,1]$: top stock $\approx 1$, bottom stock $\approx 0$.
\end{itemize}
\end{frame}

\begin{frame}{Comparison of Targets}
\begin{center}
\begin{tabular}{lcc}
& \textbf{Raw Returns} & \textbf{Ranked Returns} \\
\hline
Scale & varies by month & always $[0,1]$ \\
Outliers & can be extreme & bounded \\
Market effect & included & removed \\
Interpretation & absolute return & relative ranking \\
Loss function & MSE on returns & MSE on ranks \\
\end{tabular}
\end{center}

\vspace{0.3cm}
Using ranked returns is common in quantitative equity research.  It focuses the model on what actually drives portfolio construction: \alert{selecting the right stocks relative to their peers}.
\end{frame}

%%% PART 3: MODEL REVIEW %%%

\begin{frame}[c]
\centering
\Huge Models: A Review
\end{frame}

\begin{frame}{The Prediction Problem}
We have a panel of stock-months.  For each stock $i$ in month $t$:
\begin{itemize}
\item Features: $\mathbf{x}_{i,t-1}$ (known at the start of month $t$)
\item Target: $y_{i,t}$ (return or return rank during month $t$)
\end{itemize}

\vspace{0.3cm}
We want to learn a function $f$ such that
\[
y_{i,t} \approx f(\mathbf{x}_{i,t-1})
\]

\vspace{0.3cm}
\begin{itemize}
\item Train on past data, predict on future data (no look-ahead bias).
\item The function $f$ can be linear, a tree ensemble, a neural network, etc.
\end{itemize}
\end{frame}

%%% LINEAR MODELS %%%

\begin{frame}{Linear Models}
\[
f(\mathbf{x}) = \beta_0 + \beta_1 x_1 + \cdots + \beta_k x_k
\]

\begin{itemize}
\item Simple, interpretable, fast to estimate.
\item Each coefficient $\beta_j$ measures the marginal effect of feature $j$.
\item Assumes the effect of each feature is \alert{additive and linear} --- no interactions, no diminishing effects.
\end{itemize}

\vspace{0.3cm}
\textbf{Regularization:}
\begin{itemize}
\item \textbf{Ridge} ($L_2$): shrinks coefficients toward zero, keeps all features.
\item \textbf{Lasso} ($L_1$): drives some coefficients to exactly zero (feature selection).
\item \textbf{Elastic net}: combines $L_1$ and $L_2$ penalties.
\end{itemize}
\end{frame}

\begin{frame}{Limitations of Linearity}
\begin{itemize}
\item A linear model says: ``if momentum goes up by 1, expected return goes up by $\beta$.''
\item But the real relationship may be \alert{nonlinear}:
\begin{itemize}
\item Momentum may help only above a threshold.
\item Value may matter more for small stocks than large stocks (interaction).
\item The effect of leverage may reverse at high levels.
\end{itemize}
\item We need models that can capture these patterns automatically.
\end{itemize}
\end{frame}

%%% TREE MODELS %%%

\begin{frame}{Decision Trees}
A decision tree partitions the feature space into rectangular regions using a sequence of binary splits.

\vspace{0.3cm}
\begin{itemize}
\item At each node, pick the feature and threshold that best reduces prediction error.
\item Naturally captures \alert{nonlinearities} and \alert{interactions}.
\item A single tree is interpretable but tends to overfit.
\end{itemize}

\vspace{0.3cm}
\textbf{Solution:} combine many trees into an \alert{ensemble}.
\end{frame}

\begin{frame}{Random Forest}
\begin{itemize}
\item Grow $B$ trees, each on a \alert{bootstrap sample} of the training data.
\item At each split, consider only a \alert{random subset} of features.
\item Prediction = average of all $B$ trees.
\end{itemize}

\vspace{0.3cm}
\textbf{Key properties:}
\begin{itemize}
\item Reduces variance through averaging (bagging).
\item Random feature selection decorrelates trees.
\item Robust to outliers, requires little tuning, and hard to overfit by adding more trees.
\end{itemize}
\end{frame}

\begin{frame}{Gradient Boosting (GBM)}
Instead of averaging independent trees, build trees \alert{sequentially}, each correcting the errors of the previous ones.

\vspace{0.3cm}
\begin{enumerate}
\item Start with a constant prediction (e.g., the mean).
\item Fit a shallow tree to the \alert{residuals} (prediction errors).
\item Add the new tree's prediction (scaled by learning rate $\eta$) to the running total; repeat for $B$ rounds.
\end{enumerate}

\vspace{0.3cm}
\textbf{Key hyperparameters:} number of trees $B$, learning rate $\eta$, tree depth, minimum samples per leaf.
\end{frame}

\begin{frame}{LightGBM}
\begin{itemize}
\item Microsoft's implementation of gradient boosting, optimized for speed and scale.
\item Uses \alert{histogram-based} splitting and \alert{leaf-wise} growth for faster, more accurate tree building.
\item Widely used in quantitative finance and Kaggle competitions.
\end{itemize}

\vspace{0.3cm}
In practice, LightGBM often outperforms random forests on tabular financial data.
\end{frame}

%%% RANDOM FOURIER FEATURES %%%

\begin{frame}{Random Fourier Features}
A way to add nonlinearity to a linear model without the complexity of trees or neural networks.
\vspace{0.3cm}

\textbf{Idea:} transform the original $k$ features into a much larger set of $D$ random nonlinear features, then fit a linear model on the transformed features.

\[
\phi(\mathbf{x}) = \sqrt{\frac{2}{D}} \begin{pmatrix} \cos(\mathbf{w}_1'\mathbf{x} + b_1) \\ \vdots \\ \cos(\mathbf{w}_D'\mathbf{x} + b_D) \end{pmatrix}
\]

where $\mathbf{w}_j \sim N(0, \gamma^2 I)$ and $b_j \sim \text{Uniform}(0, 2\pi)$ are drawn \alert{randomly and fixed}.
\end{frame}

\begin{frame}{Why Random Fourier Features Work}
\begin{itemize}
\item The cosine transformation creates \alert{nonlinear combinations} of the original features, capturing interactions and curved relationships.
\item The random weights $\mathbf{w}_j$ are not learned --- only the final linear coefficients are estimated.
\item Approximates a \alert{kernel} method (RBF kernel) that scales to large datasets; bandwidth $\gamma$ controls the ``wiggliness'' of the fit.
\end{itemize}

\vspace{0.3cm}
\textbf{Advantages:} fast to fit (just linear regression on transformed features), no neural network training, easy to regularize with ridge or lasso.
\end{frame}

%%% NEURAL NETWORKS %%%

\begin{frame}{Neural Networks}
A neural network is a sequence of linear transformations interleaved with nonlinear \alert{activation functions}.
\vspace{0.3cm}

For a network with one hidden layer of $h$ units:
\[
f(\mathbf{x}) = \mathbf{w}_2' \,\sigma\!\bigl(\mathbf{W}_1 \mathbf{x} + \mathbf{b}_1\bigr) + b_2
\]

\begin{itemize}
\item $\mathbf{W}_1$ is an $h \times k$ weight matrix, $\sigma(\cdot)$ is a nonlinear activation (ReLU, tanh, etc.).
\item The hidden layer learns \alert{nonlinear features} from the raw inputs.
\item Adding more layers = deeper network = more complex representations.
\end{itemize}
\end{frame}

\begin{frame}{Training Neural Networks}
\begin{itemize}
\item Minimize a loss function (e.g., MSE) using \alert{stochastic gradient descent} (SGD) or variants like Adam.
\item Process data in \alert{mini-batches}; one pass through the full training set = one \alert{epoch}.
\item Training continues for many epochs; monitor validation loss to detect overfitting.
\end{itemize}

\vspace{0.3cm}
\textbf{Regularization:}
\begin{itemize}
\item \textbf{Dropout:} randomly zero out a fraction of hidden units during training.
\item \textbf{Early stopping:} stop training when validation loss stops improving.
\item \textbf{Weight decay} ($L_2$ penalty on weights).
\end{itemize}
\end{frame}

\begin{frame}{Neural Networks vs.\ Other Models}
\begin{center}
\begin{tabular}{lcccc}
& \textbf{Linear} & \textbf{Trees} & \textbf{RFF} & \textbf{Neural Net} \\
\hline
Nonlinearity & no & yes & yes & yes \\
Interactions & no & yes & yes & yes \\
Interpretable & yes & moderate & no & no \\
Tuning effort & low & moderate & low & high \\
Training speed & fast & fast & fast & slow \\
\end{tabular}
\end{center}

\vspace{0.3cm}
No single model dominates.  In practice, \alert{ensembles} that combine predictions from multiple model types often perform best.
\end{frame}

\end{document}
