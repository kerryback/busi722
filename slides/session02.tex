\documentclass[aspectratio=169]{beamer}
\usetheme{metropolis}

\usepackage{graphicx}
\usepackage{hyperref}
\usepackage{amsmath}
% MGMT 675 Beamer Style
% Include this file in your slide decks with: \input{mgmt675-style}

\usepackage{tcolorbox}
\tcbuselibrary{skins}
\usepackage{tikz}
\usepackage{booktabs}
\usetikzlibrary{shadings}

% Spacing adjustments
\linespread{0.9}  % Tighter baseline skip
\setlength{\parskip}{0.8\baselineskip plus 0.3ex}  % Extra space between paragraphs
\setbeamertemplate{itemize items}[circle]
\setlength{\leftmargini}{1.5em}

% Extra space between bullet items
\let\olditemize\itemize
\renewcommand{\itemize}{%
  \olditemize
  \setlength{\itemsep}{0.7ex}%
  \setlength{\parsep}{0.3ex}%
}

% Define accent color (matches title underline)
\definecolor{accentblue}{RGB}{37,99,235}
% Light blue/teal for shaded boxes
\definecolor{excelinput}{RGB}{232, 244, 247} %{177,214,224}
% Dark teal for titles
\definecolor{titlegray}{RGB}{19,60,71}

% Title slide highlight line color
\definecolor{titlehighlight}{RGB}{9,112,140}
\setbeamercolor{progress bar}{fg=titlehighlight}

% Alert text styling - bold bright coral/orange for contrast
\definecolor{alertorange}{RGB}{230,90,50}
\setbeamercolor{alerted text}{fg=alertorange}
\setbeamerfont{alerted text}{series=\bfseries}

% Hyperlink colors
\hypersetup{
  colorlinks=true,
  linkcolor=titlegray,
  urlcolor=alertorange,
  citecolor=accentblue
}

% Set frame title background
\setbeamercolor{frametitle}{bg=titlegray, fg=white}

% Style block environments with background shading
\setbeamercolor{block title}{fg=white, bg=accentblue}
\setbeamercolor{block body}{fg=black, bg=gray!10}

\setbeamercolor{block title alerted}{fg=white, bg=red!70!black}
\setbeamercolor{block body alerted}{fg=black, bg=red!10}

\setbeamercolor{block title example}{fg=white, bg=green!50!black}
\setbeamercolor{block body example}{fg=black, bg=green!10}

% Theorem environments
\setbeamercolor{theorem text}{fg=black}
\setbeamercolor{theoremblock title}{fg=white, bg=accentblue!80}
\setbeamercolor{theoremblock body}{fg=black, bg=accentblue!10}

% Bulleted list in shaded box
\newenvironment{baritemize}{%
  \begin{tcolorbox}[
    colback=excelinput,
    colframe=accentblue,
    boxrule=0.5pt,
    arc=2mm,
    auto outer arc,
    left=3mm,
    right=3mm,
    top=2mm,
    bottom=2mm
  ]
  \begin{itemize}
}{%
  \end{itemize}
  \end{tcolorbox}
}

% Numbered list in shaded box
\newenvironment{barenumerate}{%
  \begin{tcolorbox}[
    colback=excelinput,
    colframe=accentblue,
    boxrule=0.5pt,
    arc=2mm,
    auto outer arc,
    left=3mm,
    right=3mm,
    top=2mm,
    bottom=2mm
  ]
  \begin{enumerate}
}{%
  \end{enumerate}
  \end{tcolorbox}
}

% Define shaded box environment with slightly rounded corners and dark title bar
\newtcolorbox{shadedbox}[1][]{
  colback=excelinput,
  colframe=accentblue,
  boxrule=0.5pt,
  arc=2mm,
  auto outer arc,
  left=3mm,
  right=3mm,
  top=2mm,
  bottom=2mm,
  after skip=\baselineskip,
  coltitle=white,
  colbacktitle=titlegray,
  fonttitle=\bfseries,
  toptitle=2mm,
  bottomtitle=2mm,
  #1
}

% Gradient background for title slide
\let\oldmaketitle\maketitle
\renewcommand{\maketitle}{%
  {%
    \setbeamertemplate{background}{%
      \begin{tikzpicture}[remember picture,overlay]
        \fill[white] (current page.north west) rectangle (current page.south east);
        \shade[inner color=white, outer color=titlegray!40, shading=radial]
          (current page.north west) circle (28cm);
      \end{tikzpicture}%
    }%
    \oldmaketitle
  }%
}


\title{Claude Skills, Fundamental Indicators}
\subtitle{BUSI 722: Data-Driven Finance II}
\author{Kerry Back}
\institute{}
\date{}

\begin{document}

\maketitle

%%% PART 1: SETUP & DATABASE ACCESS %%%

\begin{frame}[c]
\centering
\Huge Database Access
\end{frame}

\begin{frame}{The Rice Data Portal}
\begin{baritemize}
\item We will use the \href{https://data-portal.rice-business.org}{Rice Data Portal} for stock market data.
\item The portal provides daily prices, valuation metrics, and fundamental data from SEC filings.
\item Four main tables:
\begin{itemize}
\item \textbf{SEP} -- daily stock prices (open, high, low, close, volume)
\item \textbf{DAILY} -- daily valuation metrics (marketcap, pb, pe, ps)
\item \textbf{SF1} -- fundamentals from 10-K/10-Q filings
\item \textbf{TICKERS} -- company metadata (sector, industry)
\end{itemize}
\end{baritemize}
\end{frame}

\begin{frame}{Storing Your Access Token}
Visit \href{https://data-portal.rice-business.org}{data-portal.rice-business.org} to get an access token.
\vspace{0.3cm}

\alert{Prompt:} ``Create a \texttt{.env} file with my Rice access token: \texttt{abc123...}''
\vspace{0.3cm}

Claude creates a \texttt{.env} file:
\begin{shadedbox}
\texttt{RICE\_ACCESS\_TOKEN=abc123...}
\end{shadedbox}

\begin{itemize}
\item The \texttt{.env} file keeps your token out of your code.
\item Python's \texttt{dotenv} library loads it automatically.
\item Add \texttt{.env} to \texttt{.gitignore} to avoid sharing your token.
\end{itemize}
\end{frame}

%%% PART 2: FETCH RETURNS %%%

\begin{frame}[c]
\centering
\Huge Step 1: Fetch Monthly Returns
\end{frame}

\begin{frame}{Fetching Returns}
\alert{Prompt:} ``Get monthly returns for all stocks from 2010 onward and save as \texttt{monthly.parquet}.''
\vspace{0.3cm}

Claude uses the \texttt{fetch-returns} skill, which:
\begin{barenumerate}
\item Queries \textbf{SEP} for end-of-month prices (year by year to avoid timeouts)
\item Queries \textbf{DAILY} for end-of-month market cap
\item Queries \textbf{TICKERS} for sector and industry
\item Calculates return, momentum, and lagged return
\item Assigns size categories (Nano- through Mega-Cap)
\end{barenumerate}

\vspace{0.3cm}
Includes \alert{delisted stocks} by default to avoid survivorship bias.
\end{frame}

\begin{frame}[fragile]{Returns Data: What You Get}
\small
\begin{verbatim}
>>> df = pd.read_parquet("monthly.parquet")
>>> df.shape
(544262, 8)
>>> df.columns
['ticker', 'month', 'return', 'momentum', 'lagged_return',
 'close', 'marketcap', 'pb']
\end{verbatim}
\vspace{0.2cm}
\begin{verbatim}
>>> df[df.ticker=="AAPL"].head(5)
ticker   month  return  momentum  lagged_return  close  marketcap   pb
  AAPL 2010-02  0.0655       NaN            NaN  6.859   174151.7  4.9
  AAPL 2010-03  0.1484       NaN         0.0655  7.308   185551.9  5.2
  AAPL 2010-04  0.1110       NaN         0.1484  8.393   213100.4  6.0
  AAPL 2010-05 -0.0161       NaN        -0.0161  9.325   237584.9  6.0
  AAPL 2010-06 -0.0208       NaN        -0.0161  9.174   233737.7  5.9
\end{verbatim}
\end{frame}

\begin{frame}{How Momentum Is Calculated}
Momentum is the cumulative return from month $t{-}13$ to month $t{-}2$:
\[
\text{momentum}_t \;=\; \frac{\text{closeadj}_{t-2}}{\text{closeadj}_{t-13}} - 1
\]

\begin{baritemize}
\item Skips the most recent month ($t{-}1$): short-term reversal contaminates the signal.
\item Uses split- and dividend-adjusted prices (\texttt{closeadj}).
\item Requires 13 months of price history, so the first 12 months are \texttt{NaN}.
\item All calculations are \alert{grouped by ticker} --- never mixing prices across stocks.
\end{baritemize}
\end{frame}

\begin{frame}{Variables from the Returns Step}
\begin{baritemize}
\item \texttt{ticker}, \texttt{month} -- identifiers
\item \texttt{return} -- monthly return (decimal: 0.05 = 5\%)
\item \texttt{momentum} -- 12-month return skipping the most recent month
\item \texttt{lagged\_return} -- prior month's return
\item \texttt{close} -- split-adjusted closing price (end of month)
\item \texttt{marketcap} -- market cap in thousands (\alert{shifted by 1 month})
\item \texttt{size} -- Nano/Micro/Small/Mid/Large/Mega-Cap
\item \texttt{sector}, \texttt{industry} -- classifications
\end{baritemize}
\end{frame}

%%% PART 3: FETCH FUNDAMENTALS %%%

\begin{frame}[c]
\centering
\Huge Step 2: Fetch Fundamentals
\end{frame}

\begin{frame}{Precomputed vs.\ Calculated Variables}
\alert{Rule:} Before calculating any ratio, check if it already exists in the database.
\vspace{0.3cm}

\textbf{Precomputed in DAILY:} \texttt{marketcap}, \texttt{pb}, \texttt{pe}, \texttt{ps}, \texttt{ev}, \texttt{evebit}, \texttt{evebitda}
\vspace{0.2cm}

\textbf{Precomputed in SF1:} \texttt{roe}, \texttt{roa}, \texttt{grossmargin}, \texttt{netmargin}, \texttt{de} (leverage), \texttt{assetturnover}, \texttt{currentratio}
\vspace{0.3cm}

\textbf{Must calculate:}
\begin{itemize}
\item Growth rates: asset growth, revenue growth
\item Custom ratios: gross profit / assets, book-to-market
\end{itemize}
\end{frame}

\begin{frame}{Fetching Fundamental Data}
\alert{Prompt:} ``Get annual fundamentals from SF1: equity, assets, gross profit, and the precomputed ratios roe, grossmargin, assetturnover, and de.  Calculate asset growth and gross-profit-to-assets.  Save as \texttt{fundamentals.parquet}.''
\vspace{0.3cm}

Claude uses the \texttt{fetch-fundamentals} skill, which:
\begin{barenumerate}
\item Checks which variables are precomputed (fetches, doesn't calculate)
\item Queries SF1 for annual data (\texttt{dimension='ARY'})
\item Calculates growth rates \alert{before} merging (critical!)
\item Queries DAILY for valuation ratios (pb, pe) if requested
\item Shifts DAILY variables by 1 month
\end{barenumerate}
\end{frame}

\begin{frame}[fragile]{Why Calculate Growth Rates Before Merging?}
After merging, fundamentals are \alert{forward-filled}: each filing's values repeat every month until the next filing.
\vspace{0.3cm}

\begin{shadedbox}[title=The Problem]
If you calculate \texttt{pct\_change()} after forward-fill, consecutive months with the same value produce \textbf{zero growth} --- which is wrong.
\end{shadedbox}

\vspace{0.3cm}
\textbf{Solution:} Calculate growth from the raw SF1 data (one row per filing), where consecutive rows are consecutive filings.

\begin{verbatim}
  df_fund['asset_growth'] = df_fund.groupby('ticker')
                              ['assets'].pct_change()
\end{verbatim}
\end{frame}

%%% PART 4: MERGE DATA %%%

\begin{frame}[c]
\centering
\Huge Step 3: Merge Returns \& Fundamentals
\end{frame}

\begin{frame}{Merging the Datasets}
\alert{Prompt:} ``Merge \texttt{monthly.parquet} with \texttt{fundamentals.parquet} and save as \texttt{merged.parquet}.''
\vspace{0.3cm}

Claude uses the \texttt{merge-data} skill, which:
\begin{barenumerate}
\item Aligns fundamentals to the first month \alert{after} the SEC filing date
\item Merges on (ticker, month)
\item Forward-fills fundamentals within each ticker
\item Shifts all SF1 variables by 1 month (look-ahead bias)
\item Shifts \texttt{close} to represent the prior month's price
\item Applies data quality filters
\end{barenumerate}
\end{frame}

\begin{frame}[fragile]{Merged Data: What You Get}
\small
\begin{verbatim}
>>> df = pd.read_parquet("merged.parquet")
>>> df.shape
(886460, 20)
>>> df[df.ticker=="AAPL"].dropna().head(5)
ticker   month  return  momentum  close  marketcap   pb    roe  grossmargin
  AAPL 2012-02  0.1883    0.1936 16.303   425612.0  4.7  0.396      0.405
  AAPL 2012-03  0.1053    0.2924 19.373   505758.5  5.6  0.396      0.405
  AAPL 2012-04 -0.0260    0.5564 21.413   559015.5  6.2  0.396      0.405
  AAPL 2012-05 -0.0107    0.7123 20.857   546072.5  5.3  0.396      0.405
  AAPL 2012-06  0.0109    0.6789 20.633   540207.8  5.3  0.396      0.405
\end{verbatim}
\vspace{0.2cm}
Note: \texttt{roe} and \texttt{grossmargin} are constant across months --- they reflect the most recent annual filing, forward-filled and shifted.
\end{frame}

%%% PART 5: LOOK-AHEAD BIAS %%%

\begin{frame}[c]
\centering
\Huge Avoiding Look-Ahead Bias
\end{frame}

\begin{frame}{The Core Problem}
We want each row to represent: \alert{what we knew at the start of the month} paired with \alert{what happened during the month}.
\vspace{0.3cm}

\begin{shadedbox}[title=Look-Ahead Bias]
Using information that was \textbf{not yet available} at the time of the investment decision. Backtests that suffer from this overstate performance --- sometimes dramatically.
\end{shadedbox}
\end{frame}

\begin{frame}{How the Skills Handle It}
\begin{baritemize}
\item \textbf{Market cap and DAILY ratios} (pb, pe, ps): shifted by 1 month after fetching.  The value for January is from end of December.
\item \textbf{Close price}: shifted by 1 month after merging.  Represents prior month's closing price.
\item \textbf{SF1 fundamentals} (roe, de, etc.): aligned to the first month \emph{after} the SEC filing date, then forward-filled, then shifted by 1 month.
\item \textbf{Momentum}: uses prices through month $t{-}2$, so it is known at the start of month $t$.
\item \textbf{Lagged return}: prior month's return, known at the start of the current month.
\end{baritemize}
\end{frame}

\begin{frame}{Summary of Shifts}
\begin{center}
\begin{tabular}{lll}
\textbf{Variable} & \textbf{Source} & \textbf{Shifted By} \\
\hline
marketcap, pb, pe, ps & DAILY & 1 month after fetching \\
close & SEP & 1 month after merging \\
roe, de, grossmargin, \ldots & SF1 & filing date $+$ 1 month after merge \\
momentum & SEP & built-in (uses $t{-}13$ to $t{-}2$) \\
lagged\_return & SEP & built-in (uses $t{-}1$) \\
return & SEP & not shifted (target variable) \\
\end{tabular}
\end{center}
\vspace{0.3cm}
\texttt{return} is the \alert{only variable that is not known at the start of the month}.  It is what we are trying to predict.
\end{frame}

%%% PART 6: FILTERS AND STANDARDIZATION %%%

\begin{frame}[c]
\centering
\Huge Filters \& Standardization
\end{frame}

\begin{frame}{Data Quality Filters}
The \texttt{merge-data} skill applies two filters automatically:
\begin{baritemize}
\item \textbf{Price filter:} drop rows with \texttt{close} $< \$5.00$
\item \textbf{Missing data:} drop rows with any \texttt{NaN} values
\end{baritemize}

\vspace{0.3cm}
\begin{center}
\begin{tabular}{lr}
Before filters: & 886,460 rows \quad 9,375 tickers \\
After filters: & 589,006 rows \quad 6,825 tickers \\
\end{tabular}
\end{center}
\end{frame}

\begin{frame}{Penny Stocks}
\begin{itemize}
\item Must filter out low-price stocks.  Infeasible for equally weighted portfolios and distort portfolio returns.
\item Use CQA Competition filter: \$5.60?
\item Example: Tell Claude to read the data and drop all rows with close $\le$ price threshold.  Sort into quintiles each month on momentum and lagret and compute average returns.
\end{itemize}
\end{frame}

\begin{frame}{Market Cap Filters}
\begin{itemize}
\item If we were managing a fund, we would specify the universe of stocks in our prospectus.
\item We would likely include some market cap filter: large cap = S\&P 500, large and midcap = Russell 1000, smallcap = Russell 2000, etc.
\item It is hard for large funds to trade microcaps, so they will exclude them.
\end{itemize}
\end{frame}

\begin{frame}{Standardizing Features}
\begin{itemize}
\item Many models are highly affected by outliers.
\item Standardizing predictive variables is common.  For example, subtract mean and divide by standard deviation (z-score).
\item Ranking 1 through $n$ is another way to standardize.  Can scale to 0 to 1 or compute z-scores of ranks.
\item We should do this each month rather than for the full sample.
\item For ``good'' features (like football scores), rank from low to high.  For ``bad'' features (like golf scores), rank from high to low.  Then a high rank is always good.
\end{itemize}
\end{frame}

\begin{frame}{Composite Ranks}
The Quality Minus Junk paper computes quality as follows:

\begin{barenumerate}
\item Compute z-scores of features each month.
\item Group features into three categories: profitability, growth, and safety.
\item Within each category, add z-scores and compute the z-score of the sum each month.
\item Add the three category z-scores and compute the z-score of the sum each month.
\end{barenumerate}
\end{frame}

%%% PART 7: THE COMPLETE WORKFLOW %%%

\begin{frame}[c]
\centering
\Huge The Complete Workflow
\end{frame}

\begin{frame}{End-to-End Prompts}
\begin{enumerate}
\item ``Get monthly returns for all stocks from 2010 onward.  Save as \texttt{monthly.parquet}.''
\item ``Get annual fundamentals from SF1: equity, assets, gp, and the precomputed ratios roe, grossmargin, assetturnover, de.  Calculate asset growth and gp/assets.  Save as \texttt{fundamentals.parquet}.''
\item ``Merge monthly returns with fundamentals.  Save as \texttt{merged.parquet}.''
\end{enumerate}

\vspace{0.3cm}
Three prompts produce a complete, bias-free panel dataset ready for portfolio analysis or machine learning.
\end{frame}

\begin{frame}[fragile]{The Final Dataset}
\small
\begin{verbatim}
>>> df = pd.read_parquet("merged.parquet")
>>> df.shape
(589006, 17)
>>> df.columns
['ticker', 'month', 'return', 'momentum', 'lagged_return',
 'close', 'marketcap', 'pb', 'asset_growth', 'roe',
 'gp_to_assets', 'grossmargin', 'assetturnover', 'leverage',
 'sector', 'industry', 'size']
>>> df.ticker.nunique()
6825
\end{verbatim}

\vspace{0.2cm}
589,006 stock-months $\times$ 17 variables, spanning 2011--2025, covering 6,825 stocks (including delisted).
\end{frame}

\end{document}
